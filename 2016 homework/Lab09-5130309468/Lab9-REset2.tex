\documentclass[12pt,a4paper]{article}
\usepackage{amsmath,amsthm,amssymb,bm}
\usepackage[usenames]{xcolor}
\usepackage{hyperref}
\usepackage{graphicx}
\usepackage{epstopdf}
\usepackage{ textcomp }
%\usepackage{float}


\newtheorem{theorem}{Theorem}
\newtheorem{lemma}[theorem]{Lemma}
\newtheorem{proposition}[theorem]{Proposition}
\newtheorem{corollary}[theorem]{Corollary}
\newtheorem{exercise}{Exercise}[section]
\newtheorem*{solution}{Solution}
\theoremstyle{definition}


\numberwithin{equation}{section}
\numberwithin{figure}{section}

\renewcommand{\thefootnote}{\fnsymbol{footnote}}

\newcommand{\postscript}[2]
 {\setlength{\epsfxsize}{#2\hsize}
  \centerline{\epsfbox{#1}}}

\renewcommand{\baselinestretch}{1.0}

\setlength{\oddsidemargin}{-0.365in}
\setlength{\evensidemargin}{-0.365in}
\setlength{\topmargin}{-0.3in}
\setlength{\headheight}{0in}
\setlength{\headsep}{0in}
\setlength{\textheight}{10.1in}
\setlength{\textwidth}{7in}
\makeatletter \renewenvironment{proof}[1][Proof] {\par\pushQED{\qed}\normalfont\topsep6\p@\@plus6\p@\relax\trivlist\item[\hskip\labelsep\bfseries#1\@addpunct{.}]\ignorespaces}{\popQED\endtrivlist\@endpefalse} \makeatother
\makeatletter
\renewenvironment{solution}[1][Solution] {\par\pushQED{\qed}\normalfont\topsep6\p@\@plus6\p@\relax\trivlist\item[\hskip\labelsep\bfseries#1\@addpunct{.}]\ignorespaces}{\popQED\endtrivlist\@endpefalse} \makeatother



\begin{document}
\noindent

%========================================================================
\noindent\framebox[\linewidth]{\shortstack[c]{
\Large{\textbf{Lab09-Recursively Enumerable Set(2)}}\vspace{1mm}\\
CS363-Computability Theory, Xiaofeng Gao, Spring 2016}}
\begin{center}
\footnotesize{\color{red}$*$ Please upload your assignment to FTP or submit a paper version on the next class}

\footnotesize{\color{red}$*$ If there is any problem, please contact: steinsgate@sjtu.edu.cn}

\footnotesize{\color{blue}$*$ Name: Yupeng Zhang \quad StudentId: 5130309468 \quad Email: 845113336@qq.com}
\end{center}

\begin{enumerate}
\item Suppose $A$ is an r.e.~set. Prove the following statements.
\begin{enumerate}
\item Show that the sets $\bigcup\limits_{x\in A}W_x$ and $\bigcup\limits_{x\in A}E_x$ are both r.e.

\textbf{Proof:}

$y \in \bigcup\limits_{x\in A}W_x \Leftrightarrow \exists z(z \in A)(P_z(y) \downarrow)$

So, the first set is r.e..

$y \in \bigcup\limits_{x\in A}E_x \Leftrightarrow \exists z_1 \exists z_2(z_1 \in A)(P_{z_1}(z_2) \downarrow y)$

So, the second set is r.e..

\item Show that $\bigcap\limits_{x\in A}W_x$ is not necessarily r.e. (\emph{Hint}: $\forall t \in \mathbb{N}$ let $K_t=\{x:P_x(x)\downarrow \mbox{ in t steps}\}$. Show that for any $t$, $K_t$ is recursive; moreover $K=\bigcup\limits_{t\in\mathbb{N}}K_t$ and $\overline{K}=\bigcap\limits_{t\in\mathbb{N}}\overline{K}_t$.)

\textbf{Proof:}

$\forall t \in \mathbb{N}$ let $K_t=\{x:P_x(x)\downarrow \mbox{ in t steps}\}$, the characteristic function of $K_t$ is:

$$c_{K_t} = \begin{cases} 1 &, P_x(x)\downarrow \mbox{ in t steps} \\ 0 &, \mbox{otherwise} \\ \end{cases}$$

So, $c_{K_t}$ is computable, thus $K_t$ and $\overline{K_t}$ are recursive.

Moreover, $K=\bigcup\limits_{t\in\mathbb{N}}K_t$ and $\overline{K}=\bigcap\limits_{t\in\mathbb{N}}\overline{K}_t$.

Since $\overline{K}=\bigcap\limits_{t\in\mathbb{N}}\overline{K}_t$ is not r.e.. Let $A^* = \mathbb{N}$, $\bigcap\limits_{t\in A^*}W_x$ is not r.e. So the set is not necessarily r.e..

\end{enumerate}

\item Prove that $A\subseteq\mathbb{N}^n$ is r.e.~iff $A=\varnothing$ or there is a total computable function $f:\mathbb{N}\rightarrow\mathbb{N}^n$ such that $A=Ran(\bm{f})$. (A \emph{computable function} $\bm{f}$ from $\mathbb{N}$ to $\mathbb{N}^n$ is an $n$-tuple $\bm{f}=(f_1,\ldots,f_n)$ where each $f_i$ is a unary computable function and $\bm{f}(x)=(f_1(x),\ldots,f_n(x))$.)

\textbf{Proof:}

First, we proof the $\Rightarrow$:

If $A = \varnothing $, then completed.

If $A \neq \varnothing$, then $\exists a \in A$. For $P$ to be the program to compute the partial characteristic function of $A$. We have:

$$f(x) = \begin{cases} ((x)_1,(x)_2,...(x)_n) &, P((x)_1,(x)_2,...(x)_n) \downarrow \mbox{in } (x)_0 \mbox{ steps}\\
 a &, \mbox{otherwise} \\
 \end{cases}$$
 
 So, $f$ is computable, thus $A = Ran(f)$

Then, we proof the $\Leftarrow:$

If $A = \varnothing $, then completed.

If $A \neq \varnothing$, then we assume that there is a total and computable function $f$ that $A = Ran(f)$.

Then, we have:$(x_1,x_2,...x_n) \in A \Leftrightarrow \exists y(f(y) = (x_1,x_2,...x_n))$.

Because of \textbf{Graph Theorem}, the right part is partially decidable. So,the set $A$ is r.e..


\item Suppose that $f$ is a total computable function, $A$ is a recursive set and $B$ is an r.e.set. Show that $f^{-1}(A)$ is recursive and that $f(A)$, $f(B)$ and $f^{-1}(B)$ are r.e. but not necessarily recursive. What extra information about these sets can be obtained if $f$ is a bijection?

\textbf{Proof:}

$x \in f^{-1}(A) \Leftrightarrow f(x) \in A$

So, $f^{-1}(A)$ is recursive.

$x \in f(A) \Leftrightarrow \exists y(y \in A)(f(y) = x)$

So, $f(A)$ is r.e..

$x \in f(B) \Leftrightarrow \exists y(y \in B)(f(y) = x)$

So, $f(B)$ is r.e..

$x \in f^{-1}(B) \Leftrightarrow f(x) \in B$

So, $f^{-1}(B)$ is r.e..

If $f$ is a bijection, $f(A)$ is recursive.

\item A set $D$ is the difference of r.e. sets (\emph{d.r.e.}) iff $D=A-B$ where $A,B$ are both \emph{r.e.}.
\begin{enumerate}
\item Show that the set of all \emph{d.r.e.} sets is closed under the formation of intersection.

\textbf{Proof:}

We assume that$A_1, A_2, B_1, B_2$ are all r.e. and $D_1 = A_1 - B_1$ and $D_2 = A_2 - B_2$.

So, there are computable functions $f$,$g$ that $f(A_1,A_2,B_1,B_2)$ and $g(A_1,A_2,B_1,B_2)$ are both r.e..

So $D_1 \cap D_2 = f(A_1,A_2,B_1,B_2) - g(A_1,A_2,B_1,B_2)$. So the set of all d.r.e. sets is closed under the formation of intersection.

\item Show that if $C_n = \{x \mid |W_x|=n \}$, then $C_n$ is \emph{d.r.e.} for all $n \ge 0$.

\textbf{Proof:}

We assume that $U$ is the set of all computable functions, and $C'_n =  \{x \mid |W_x| \neq n \}$.

Since $|W_x| = \sum$ \textbf{1} $(y \in W_x)$, so $C'_n$ is r.e..

So, $C_n = U - C'_n$ is d.r.e..



\end{enumerate}



\end{enumerate}

\end{document}
