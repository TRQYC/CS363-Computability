\documentclass[12pt,a4paper]{article}
\usepackage{amsmath,amsthm,amssymb}
\usepackage[usenames]{xcolor}
\usepackage[hidelinks]{hyperref}
\usepackage{graphicx}
\usepackage{epstopdf}
\usepackage{mathrsfs}
%\usepackage{float}


\newtheorem{theorem}{Theorem}
\newtheorem{lemma}[theorem]{Lemma}
\newtheorem{proposition}[theorem]{Proposition}
\newtheorem{corollary}[theorem]{Corollary}
\newtheorem{exercise}{Exercise}[section]
\newtheorem*{solution}{Solution}
\theoremstyle{definition}


\numberwithin{equation}{section}
\numberwithin{figure}{section}

\renewcommand{\thefootnote}{\fnsymbol{footnote}}

\newcommand{\postscript}[2]
 {\setlength{\epsfxsize}{#2\hsize}
  \centerline{\epsfbox{#1}}}

\renewcommand{\baselinestretch}{1.0}

\setlength{\oddsidemargin}{-0.365in}
\setlength{\evensidemargin}{-0.365in}
\setlength{\topmargin}{-0.3in}
\setlength{\headheight}{0in}
\setlength{\headsep}{0in}
\setlength{\textheight}{10.1in}
\setlength{\textwidth}{7in}
\makeatletter \renewenvironment{proof}[1][Proof] {\par\pushQED{\qed}\normalfont\topsep6\p@\@plus6\p@\relax\trivlist\item[\hskip\labelsep\bfseries#1\@addpunct{.}]\ignorespaces}{\popQED\endtrivlist\@endpefalse} \makeatother
\makeatletter
\renewenvironment{solution}[1][Solution] {\par\pushQED{\qed}\normalfont\topsep6\p@\@plus6\p@\relax\trivlist\item[\hskip\labelsep\bfseries#1\@addpunct{.}]\ignorespaces}{\popQED\endtrivlist\@endpefalse} \makeatother



\begin{document}
\noindent

%========================================================================
\noindent\framebox[\linewidth]{\shortstack[c]{
\Large{\textbf{Lab13-NPReduction}}\vspace{1mm}\\
CS363-Computability Theory, Xiaofeng Gao, Spring 2016}}
\begin{center}
\footnotesize{\color{red}$*$ Please upload your assignment to FTP or submit a paper version on the next class}

\footnotesize{\color{red}$*$ If there is any problem, please contact: steinsgate@sjtu.edu.cn}

\footnotesize{\color{blue}$*$ Name: Zhang Yupeng \quad StudentId: 5130309468 \quad Email: 845113336@qq.com}
\end{center}

\begin{enumerate}
  \item Find the certificate and certifier for the decision version of the following problems.
    \begin{enumerate}
      \item Clique: Given an undirected graph, find a subset $S$ that there is an edge connecting every pair of nodes in $S$ with maximum nodes.
      
      \textbf{Solution:}
      
 Decision version: Given an undirected graph $G = (V,E)$ and $k \geq 0$, does there exist a clique of size $\geq k$?
 
Certificate: A set of vertex $S \subseteq V$.
 
Certifier: Check if $|S|\geq k$ and $\forall v, v^{'} \in V((v^{'} =v) \vee ( \exists (v,v^{'}) \in E))$
      
      
      \item Metric k-center: Given $n$ cities with specified distances for each pair of cities as $d_{ij}$, one wants to build $k$ warehouses in different cities and minimize the maximum distance of a city to a warehouse.
      
      \textbf{Solution:}
      
      Decision version: Given $n$ cities with specified distances for each pair of cities as $d_{ij}$, one wants to build $k$ warehouses in different cities, can the maximum distance of a city to a warehouse be at most $s$?
      
Certificate: $k$ cities to build warehouses.
      
Certifier: Calculate the maximum distance of one city to one warehouse and check whether it is no more than $s$.
      
      \item Set Packing: Given a set $U$ of $n$ elements, a collection $S_1, \cdots, S_m$ of
subsets of $U$, find the maximum subsets such that no two of them intersect.

    \textbf{Solution:}
    
    Decision: Given a set $U$ of $n$ elements, a collection $S_1, ... , S_m$ of subsets of $U$, can there exists no less than $k$ subsets such that no two of them intersect.
    
Certificate: A collection of subsets $\{S_1^{'} , ... , S_j^{'} \} \subseteq \{S_1, ... , S_m\}$
    
Certifier: Check whether $\{S_1^{'} , ... , S_j^{'} \}$ contains no less than $k$ elements and whether any of them intersect.
    

	  \item minimum k-cut: Given a weighted graph $G=(V,E)$, we want to find a minimum weighted set of edges whose removal would partition the graph to $k$ connected components.
	  
	  \textbf{Solution:}
	  
	 Decision version: Given a weighted graph $G = (V,E)$, is there a weighted set of edges with total weight no more than $s$ that the removal of this set would seperate the graph to $k$ connected components.
	 
Certificate: A set of edges to remove.
	 
Certifier: Check whether the total weight of the removed edges no more than $s$ and whether they seperate the graph to $k$ connected components.
	  
    \end{enumerate}

  \item The knapsack problem is a well-known optimization problem. Given a set of $n$ items, each item $i$ with a weight $w_i$ and a value $v_i$, determine the number of each item to include in a collection so that the total weight is less than or equal to a given limit $W$ and the total value is as large as possible.

  Prove that the knapsack problem is NP-complete. {(\color{blue} Hint: One solution is reducing the Subset Sum problem to it.)}
  
  \textbf{Proof:}
  
  We'll reduce the Subset Sum problem to the knapsack problem in order to prove the problem.
  
Given an instance of Subset Sum problem with number values $n_i$ and the objective sum $S$ and construct the knapsack problem as follows:
  
   Denote the weight and the value for each item $i$ as $w_i$ and $v_i$ and the weight limit as $W$, then we have $w_i =v_i =n_i$ and $W =S$. Since the value is equal to the weight, we can know that the largest value $V_{max} = W$.
   
    Under this construction, to find a subset of numbers that adds up to $S$ is to find a collection of items that the total weight is less than $W$. Thus, we reduce the Subset Sum problem to the knapsack problem.
    
Since the Subset Sum problem is NP-complete, we can prove that the knapsack problem is NP-complete.
  
  

  \item We know that $\textbf{P} \subseteq \textbf{NP} \cap \textbf{co-NP}$. Please give an example that belongs to following set. If you can, briefly explain your reason. (Should be examples different from the course slides).
  \begin{enumerate}
    \item \textbf{Co-NP}.
    
    \textbf{Solution:}
    
    For $a,b \in \mathbb{N}$, is $a = b$?
    
Because $\textbf{P} \subseteq \textbf{Co-NP}$.
        
    \item \textbf{Co-NP} $\cap$ \textbf{NP-hard}.
    
    \textbf{Solution:}
    
    Clique.
    
    Because $\textbf{NP-complete} \in \textbf{Co-NP} \cap \textbf{NP-hard}$.
    
    
    
    \item \textbf{Co-NP} $\cap$ \textbf{NP}, but not known to be in \textbf{P}.
    
    \textbf{Solution:}
    
     Pigeonhole Subset Sum: Given $n$ positive integers with sum less than $2^n-1$, find two disjoint nonempty subsets whose sums are equal. 
     
     
  \end{enumerate}
\end{enumerate}
\end{document}
