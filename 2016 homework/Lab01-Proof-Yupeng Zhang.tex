\documentclass[12pt,a4paper]{article}

\usepackage{amsmath,amscd,amsbsy,amssymb,latexsym,url,bm,amsthm}
\usepackage{epsfig,graphicx,subfigure}
\usepackage{enumitem,balance}
\usepackage{wrapfig}
\usepackage{mathrsfs, euscript}
\usepackage[usenames]{xcolor}
\usepackage{hyperref}
\usepackage[boxed]{algorithm2e}

\newtheorem{theorem}{Theorem}[section]
\newtheorem{lemma}[theorem]{Lemma}
\newtheorem{proposition}[theorem]{Proposition}
\newtheorem{corollary}[theorem]{Corollary}
\newtheorem{exercise}{Exercise}[section]
\newtheorem*{solution}{Solution}
\theoremstyle{definition}


\numberwithin{equation}{section}
\numberwithin{figure}{section}

\renewcommand{\thefootnote}{\fnsymbol{footnote}}

\newcommand{\postscript}[2]
 {\setlength{\epsfxsize}{#2\hsize}
  \centerline{\epsfbox{#1}}}

\renewcommand{\baselinestretch}{1.0}

\setlength{\oddsidemargin}{-0.365in}
\setlength{\evensidemargin}{-0.365in}
\setlength{\topmargin}{-0.3in}
\setlength{\headheight}{0in}
\setlength{\headsep}{0in}
\setlength{\textheight}{10.1in}
\setlength{\textwidth}{7in}
\makeatletter \renewenvironment{proof}[1][Proof] {\par\pushQED{\qed}\normalfont\topsep6\p@\@plus6\p@\relax\trivlist\item[\hskip\labelsep\bfseries#1\@addpunct{.}]\ignorespaces}{\popQED\endtrivlist\@endpefalse} \makeatother
\makeatletter
\renewenvironment{solution}[1][Solution] {\par\pushQED{\qed}\normalfont\topsep6\p@\@plus6\p@\relax\trivlist\item[\hskip\labelsep\bfseries#1\@addpunct{.}]\ignorespaces}{\popQED\endtrivlist\@endpefalse} \makeatother



\begin{document}
\noindent

%========================================================================
\noindent\framebox[\linewidth]{\shortstack[c]{
\Large{\textbf{Lab01-Proof}}\vspace{1mm}\\
CS363-Computability Theory, Xiaofeng Gao, Spring 2016}}
\begin{center}
\footnotesize{\color{red}$*$ Please upload your assignment to TA's FTP. Contact nongeek.zv@gmail.com for any questions.}

\footnotesize{\color{blue}$*$ Name: Zhang Yupeng \quad StudentId: 5130309468 \quad Email: 845113336@qq.com}
\end{center}

\begin{enumerate}

\item Prove that for any integer $n>2$, there is a prime $p$ satisfying $n<p<n!$. {\color{blue}(Hint: consider a prime factor $p$ of $n!-1$ and use proof by contradiction)}
\\ \textbf{Proof}: Assume that for any integer $n>2$,there is no prime p satisfying $n<p<n!$.
\\ The adjacent two natural numbers are co-prime, so $n!$ and $n!-1$ are co-ptime.
\\ Because $n!=1*2...*n-1*n$, so $1,2,3,...,n-1,n$ are all factors of $n!$.
\\ So $1,2,...n$ all aren't factors of $n!-1$.
\\ So the prime factor of $n!-1$ is greater than n, which contradicts our assumption.
\\ So we proof it by contradiction.



\item Use minimal counterexample principle to prove that: for every integer $n>17$, there exist integers $i_n\ge 0$ and $j_n\ge 0$, such that $n = i_n \times 4 + j_n \times 7$.
\\
\textbf{Proof}: If $n=i_n*4+j_n*7$ is not true for every integer $n>17$, then there are values of n for which $n \neq i_n*4+j_n*7$, and there must be a smallest such value, say $n=k$. 
\\
Since $18=1*4+2*7, 19=3*4+1*7, 20=5*4,21=3*7,22=2*4+2*7$, we have $k \geq 23, k-4>18$.\\
Sinve $k$ is the smallest value for which $k \neq i_k*4+j_k*7$, so $k-4 = i_{k-4} *4+j_{k-4}*7$ is true.
However, we have $k=k-4+4=i_{k-4} *4+j_{k-4}*7+4=(i_{k-4}+1)*4+j_{k-4}*7$,which derived a contradiction. So our original assumption is false.

\item Suppose $a_0=1$, $a_1=2$, $a_2=3$, $a_k=a_{k-1}+a_{k-2}+a_{k-3}$ for $k \ge 3$. Use strong principle of mathematical induction to prove that $a_n \le 2^n$ for all integers $n\ge 0$.\\
\textbf{Proof}: Obviously, $a_0 \leq 2^0, a_1 \leq 2^1, a_2 \leq 2^2$, and  $a_3=a_0+a_1+a_2=1+2+3=6<2^3$.\\
We assume that $a_n<2^n$ is true for every $n$ satisfying $n_0 \leq n \leq k, k \geq 0$.\\
Then, $a_{k+1}=a_{k}+a_{k-1}+a_{k-2} \leq 2^k+2^{k-1}+2^{k-2} <2^{k+1}$.\\
So, we proof the original assumption.


\item Consider the following loop, written in pseudocode:

\begin{center}
\begin{minipage}[b]{0.2\textwidth}
\begin{algorithm}[H]
  \While{B}{
   $S$\;
  }
\end{algorithm}
\end{minipage}
\end{center}
A condition $P$ is called an invariant of the loop if whenever $P$ and $B$ are both true, and $S$ is executed once, $P$ is still true.

\begin{enumerate}
  \item Prove that if $P$ is an invariant of the loop, and $P$ is true before the first iteration of the loop, then if the loop eventually terminates (i.e., after some number of iterations, $B$ is false), $P$ is still true.\\
  \textbf{Proof}: Because $P$ is an invariant of the loop. And $P$ is true before the first iteration of the loop, so if $B$ is true, then $S$ is excuted once, $P$ is still true.
  \\If $B$ is false, then $S$ cannot be excuted, so $P$ will maintain the value in the last iteraion. So $P$ is still true.
  \\So we can proof that when the loop terminates, $P$ is still true. 
  \item Suppose $x$ and $y$ are integer variables, and initally $x\ge 0$ and $y > 0$. Consider the following program fragment:

\begin{center}
\begin{minipage}[b]{0.25\textwidth}
\begin{algorithm}[H]
   $q$ = 0\;
   $r$ = $x$\;
   \While{$r \ge y$}{
      $q = q+1$\;
      $r = r-y$\;
   }
\end{algorithm}
\end{minipage}
\end{center}

By considering the condition ($r\ge 0) \wedge (x=q \times y+r$), prove that when this loop terminates, the values of $q$ and $r$ will be the integer quotient and remainder, respectively, when $x$ is divided by $y$; in other words, $x=q \times y+r$ and $0 \le r <y$.\\
\textbf{Proof}:We claim that the loop invariant $x$:\\
\begin{center}
	$x = qy + r $
\end{center}
Because$x=qy+r,x \geq 0,q=0,r=x$ before the loop executes. So $x$ is true before the loop.\\
Then we assume that $x$ is true before the loop is executed. Then, after the loop executes, we have the new values $r_n = r - y$ and $q_n = q + 1$.\\
Since, by the condition of the loop we know that $r \geq y$, so we have that $r_n = r - y \geq 0.$\\
Furthermore, $x = qy+r = qy+r-y+y = (qy+y)+(r-y) = (q+1)y+(r-y) = q_ny+r_n$
Thus, x is still true after the loop executes.
When the loop terminates, the condition of the loop is false, so that $r < y$.
So, $x=q*y+r$ and $0 \leq r < y$.

\end{enumerate}
\end{enumerate}
%\begin{eqnarray*}
%k& = &2*a' + 3*b' +1 \\
%& = & 2*(a'-1) + 3*(b'+1)\\
%&=& 2*(a'+2) + 3*(b'-1)
%\end{eqnarray*}
%========================================================================
\end{document}
